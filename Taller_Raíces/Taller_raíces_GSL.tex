\documentclass[spanish,10pt,letterpaper]{article}
\usepackage[T1]{fontenc}
\usepackage{babel}
\title{Taller: Comparación experimental de métodos de búsqueda de raíces (1D) con GSL}
\author{Alexander Pérez Ruiz}
\begin{document}
	\maketitle
	
	\section{Objetivos }
	
	\begin{itemize}
		\item Comparar en la práctica los métodos de búsqueda de raíces más usados (bisección, falsa posición, secante, Brent, Newton) en términos de:
		\begin{itemize}
			\item convergencia (velocidad y órdenes)
			\item robustez (sensibilidad a la elección inicial / bracketing)
			\item número de evaluaciones de $f$ y $f'$
			\item estabilidad numérica
			\item tiempo de cómputo en condiciones realistas.
		\end{itemize}
	
	\item Desarrollar habilidad para instrumentar experimentos numéricos (medir, graficar, interpretar).
	
	\item Formular conclusiones cualitativas y cuantitativas que no puedan obtenerse únicamente pidiendo respuestas a una IA (por ejemplo: comportamiento en hardware limitado, variabilidad estadística entre ejecuciones con ruido en parámetros, intuiciones sobre cuándo preferir un método en ingeniería real).
\end{itemize}
\section{Funciones a evaluar}
\begin{enumerate}
	\item $f_1(x)=x^3-6x^2+11x-6$
	\item $f_2(x)=(x-1)^2(x-2)$
	\item $f_3(x)=tanh(10(x-0.1))-0.5$
	\item $f_4(x)=sin(50x)/x$
	\item $f_5(x)=|x-0,5|-0,1$
\end{enumerate}

\section{Experimentos}
Para cada función y cada método:
\begin{itemize}
	\item Preparar: elegir intervalos/brackets o puntos iniciales razonables (documentar elección).
	\item Criterios de parada: tolerancia en $|f(x)|<$~ tol y $|\Delta x|<$~ tol. Registrar ambos.
	\item Medir:
	\begin{itemize}
		\item número de iteraciones,
		\item número de evaluaciones de $f$ y de $f'$,
		\item tiempo de CPU,
		\item si converge o no (y en cuántas iteraciones máximas — por ejemplo 200).
	\end{itemize}
	\item Variación de inicialización: repetir con varias semillas de inicialización.
\end{itemize}


\section{Preguntas orientadoras}
\begin{itemize}
	\item Para cada función, ¿qué método recomiendan en producción (por fiabilidad y coste) y por qué? (fundamentar con datos experimentales).

	\item ¿Dónde observan discrepancias entre la teoría (orden de convergencia) y los resultados prácticos? Expliquen las razones (redondeo, límites de representación, evaluaciones finitas, mala inicialización).
	
	\item ¿Qué fallos o inconsistencias inesperadas encontraron (p. ej. saltos, NaNs, convergencias a raíces falsas) y cómo los diagnosticarías y arreglarías en código de producción?
	
	\item ¿Cómo afecta el costo de evaluar $f$ (p. ej. si $f$ es una simulación costosa) la elección del método? Diseñen una regla práctica para elegir el método en función del costo de evaluación relativa de $f$ y $f'$.
	
	\item ¿Qué precauciones adicionales implementarían en una librería real (logging, límites de iteración, backoffs, switches entre métodos)? Proponer un pseudocódigo para un "root-finder robusto".
	
	\item  Si el hardware es limitado (p. ej. microcontrolador con punto flotante simple), ¿qué métodos preferirían y por qué? Comparar con resultados en PC.
	
	\item Describe un experimento de "falsificación" que mostraría que un resultado empírico tuyo no es un artefacto de una mala instrumentación o azar.
\end{itemize}
\end{document}